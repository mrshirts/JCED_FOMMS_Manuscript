\documentclass[11pt,a4paper]{article}
\usepackage{graphicx}
% uncomment according to your operating system:
% ------------------------------------------------
\usepackage[latin1]{inputenc}    %% european characters can be used (Windows, old Linux)
%\usepackage[utf8]{inputenc}     %% european characters can be used (Linux)
%\usepackage[applemac]{inputenc} %% european characters can be used (Mac OS)
% ------------------------------------------------
\usepackage{authblk}
\usepackage[superscript]{cite}
\usepackage[document]{ragged2e}
\usepackage[T1]{fontenc}   %% get hyphenation and accented letters right
\usepackage{mathptmx}      %% use fitting times fonts also in formulas
% do not change these lines:
\pagestyle{empty}                %% no page numbers!
\usepackage[left=35mm, right=35mm, top=15mm, bottom=20mm, noheadfoot]{geometry}
%% please don't change geometry settings!

\usepackage{fullpage}
\usepackage{amsfonts}
\usepackage{graphicx}
\usepackage{float}
\usepackage{amsmath}
\usepackage{chemfig}
\usepackage{indentfirst}
\usepackage{longtable}
\usepackage{array}
\usepackage{cellspace}
\usepackage{palatino}
%\usepackage{breqn}
\usepackage{amssymb}
\usepackage{verbatim}
\usepackage[colorlinks=true,citecolor=blue,linkcolor=blue]{hyperref}
\usepackage{siunitx}
\usepackage{xr}

% italicized boldface for math (e.g. vectors)
\newcommand{\bfv}[1]{{\mbox{\boldmath{$#1$}}}}
% non-italicized boldface for math (e.g. matrices)
\newcommand{\bfm}[1]{{\bf #1}}          

%\newcommand{\bfm}[1]{{\mbox{\boldmath{$#1$}}}}
%\newcommand{\bfm}[1]{{\bf #1}}
\newcommand{\expect}[1]{\left \langle #1 \right \rangle} % <.> for denoting expectations over realizations of an experiment or thermal averages

\newcommand{\var}[1]{{\mathrm var}{(#1)}}
\newcommand{\x}{\bfv{x}}
\newcommand{\y}{\bfv{y}}
\newcommand{\f}{\bfv{f}}

\newcommand{\hatf}{\hat{f}}

\newcommand{\bTheta}{\bfm{\Theta}}
\newcommand{\btheta}{\bfm{\theta}}
\newcommand{\bhatf}{\bfm{\hat{f}}}
\newcommand{\Cov}[1] {\mathrm{cov}\left( #1 \right)}
\newcommand{\T}{\mathrm{T}}                                % T used in matrix transpose

\newcommand\blfootnote[1]{%
	\begingroup
	\renewcommand\thefootnote{}\footnote{#1}%
	\addtocounter{footnote}{-1}%
	\endgroup
}


% begin the document
\begin{document}
	\thispagestyle{empty}
	%make title bold and 14 pt font (Latex default is non-bold, 16 pt)
	\title{\Large \textbf{Configurational reweighting...}}
	\author[1]{\large {\underline{Richard Messerly}}}%%[12 pt regular, presenting speaker underlined]
	
	\affil[1]{\textit{Thermodynamics Research Center (TRC), National Institute of Standards and Technology (NIST),
			Boulder, Colorado, 80305, USA}}
	
	\date{} % <--- leave date empty
	\maketitle\thispagestyle{empty} %% <-- you need this for the first page
	\begin{center}
		\title{\textbf{ABSTRACT}}\centering{}
	\end{center}
	\justify
	
\section*{Key points}

MBAR is equivalent to standard histogram reweighting approaches in the limit of zero bin width
MBAR allows for estimating VLE properties of any force field
Scaling epsilon values post-simulation is straightforward
Basis functions are an efficient means for re-computing energy


\section*{Outline}

\section{Introduction}

\begin{enumerate}
	\item History of reweighting methods in simulation
	\item Importance of phase coexistence in force field parameterization
	\item Gross demonstrated that for non-transferable parameter sets it is usually sufficient to just scale epsilon
\end{enumerate}

\section{Methods}

\subsection{Simulation set-up}

\begin{enumerate}
	\item GCMC simulations in GOMC
	\item Simulation specifications, i.e., box size, number of steps, type of moves, etc.
\end{enumerate}

\subsection{Multistate Bennett Acceptance Ratio}

\begin{enumerate}
	\item MBAR equations
	\item MBAR is equivalent to histogram reweighting in the limit of zero bin width
\end{enumerate}

\subsection{Basis functions}

\begin{enumerate}
	\item When applying MBAR to different parameter sets, it is necessary to recompute U
	\item Basis functions accelerate the recompute energy step
\end{enumerate}

\section{Results}

\begin{enumerate}
	\item Validation that MBAR and HR are indistinguishable
	\begin{enumerate}
		\item Evaluate all of the compounds that Mohammad has U and N values for
		\item Compare MBAR results with either Potoff's or my own HR results (might be better to use my own for self consistency)
	\end{enumerate}
    \item Validation of the basis function approach
    \item Epsilon scaling for all the compounds that Mohammad has U and N values for
    \item Report basis functions for several molecules with TraPPE and Potoff force fields
\end{enumerate}

Figures:

\begin{enumerate}
	\item Percent deviation between MBAR and HR results for rholiq, rhovap, and Psat
	\item Validation that basis functions give accurate energies
	\item Scaling of epsilon post-simulation for several molecules (would probably want to use same scoring function)
	\item 
\end{enumerate}

\section{Discussion/Limitations}

\begin{enumerate}

\end{enumerate}

\section{Conclusions}

\section{Acknowledgments}

\section{Supporting Information}

\end{document}
